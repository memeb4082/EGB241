\documentclass{book/custombook}

\unitname{Electromagnetics and Machines}
\unitcode{EGB241}
\unitcoordinator{Jacob Coetzee}
\author{Dinal Atapattu}

\begin{document}
    \maketitle
    \tableofcontents
    \chapter{Alternating Current}
        \section{AC Signals}
            \begin{itemize}
                \item Direct Current: DC has polarity (direction) which stays the same. Amplitude may vary, but
                    charge always flows in the same direction
                \item Alternating Current: Voltage polarity and current direction reverses periodically
                \item Period: Length of time (seconds) for one repetition of a cycle
                \item Frequency: $f = \frac{1}{T}$ [Hz]
            \end{itemize}
        \section{Sine Wave}
           In its purest form, AC signals are sine waves, for example:\\
            $ v(t) = V_o \sin\left(2\pi ft\right) = V_o\sin\left(\omega t\right) $
            Any AC wave that consists of only one frequency is sinusoidal\\
            \textcolor{blue}{Utility AC is a sine wave with frequency 50Hz}
        \section{Amplitude of an AC wave}
            \begin{itemize}
                \item \textcolor{black}{Peak Amplitude} ($V_p$): Maximum positive deviation
                \item \textcolor{black}{Peak-to-Peak Amplitude} ($V_{p-p}$): Net difference between positive
                    and negative peak amplitude
                    \begin{figure}[H]
                        \centering
                        \begin{equation}V_{p-p} = 2\times V_p\end{equation}
                        \caption{Peak to Peak Amplitude}
                    \end{figure}
                \item \textcolor{black}{Root-Mean-Square Amplitude} ($V_{rms}$): Effective amplitude of a sinusoidal AC
                    wave. Average power dissipated in a resistor with an AC voltage = power dissipated with DC voltage of
                    $V_{rms}$\\
                    \textcolor{red}{$V_{rms} \leq V_{p} \forall \text{ AC Waves}$}
                    \begin{figure}[H]
                        \centering
                        \begin{equation}
                            V_{rms} = \sqrt{\frac{1}{T} \int^{T}_{0} v^{2}(t) dt} 
                                    = \sqrt{\frac{1}{T} \int^{T+\alpha}_{\alpha} v^{2} (t) dt}
                        \end{equation}
                        \caption{RMS Amplitude}
                    \end{figure}
            \end{itemize}
            \begin{figure}[H]
                \centering
                \begin{tabular}{|c|c|}
                    \hline
                    Sinusoidal AC Wave & $V_{rms} = \frac{V_o}{\sqrt{2}} \approx 0.707V_o$\\
                    \hline
                    Square Ac Wave & $V_{rms} = V_p$\\
                    \hline
                    Triangular Wave & $V_{rms} = \frac{V_o}{\sqrt{3}} \approx 0.577V_o$\\
                    \hline
                \end{tabular}
                \caption{AC RMS Values}
            \end{figure}
        \section{Phase Shifting}
            $f(x+\alpha)$ is a horizontal translation of $f(x)$\\
            Where:
            \begin{itemize}
                If $ a > 0$, $f(x)$ is shifted to the left\\
                If $ a < 0$, $f(x)$ is shifted to the right
            \end{itemize}
            If $v_1 (t) = V_o \sin(\omega t)$, then $v_2 = V_o\sin(\omega t + \phi) = 
            V_o \sin\left[\omega\left(t + \frac{\phi}{\omega}\right)\right]$ is a shifted version of $v_1(t)$
            by $\frac{\phi}{\omega}$ seconds to the left (where $\phi$ is in radians and positive$)\\
            $v_2(t)$ leads $v_1(t)$ by $\phi$ radians ($180\times \frac{\phi}{\pi}$ degrees)\\
            $v_1(t)$ lags $v_2(t)$ by $\phi$ radians ($180\times \frac{\phi}{\pi}$ degrees)
        \section{Complex Numbers}
            \subsection{Polar Form}
                \begin{align*}
                    \text{Rectangular form}: &= z = x+jy\\
                    \text{Polar form}: &= z = re^{j\theta} = r\angle \theta\\
                \end{align*}
                \begin{itemize}
                    \item r is called the \textbf{magnitude} of $z: r=|z| = \sqrt{x^2 + y^2}$
                    \item $\theta$ is called the \textbf{argument} of $z: \theta = \atan(\frac{y}{x})$ or 
                        more accurately $\theta = \text{arg}(x+jy)$, $-\pi < \theta \leq \pi $ radians (Convention)
                \end{itemize}
            \subsection{Properties of Complex Numbers}
                \begin{figure}[H]
                    \begin{subfigure}{\linewidth}
                        \begin{align*}
                            (x+jy)\pm(p+jq) = (x\pm p) + j(y\pm q)\\ \tag{add real and imaginary parts}
                        \end{align*}
                        \caption{Addition and Subtraction}
                    \end{subfigure}
                    \begin{subfigure}{\linewidth}
                        \begin{align*}
                            (x+jy)\times(p+jq) = (xp - yq) + j(yp + xq)\\
                            \intertext{\text{Or in polar form}}
                            Ae^{j\theta} \times Be^{j\phi} = ABe^{j(\theta + \phi)}\\ \tag{multiply magnitudes, add angles}
                        \end{align*}
                        \caption{Multiplication}
                    \end{subfigure}
                    \begin{subfigure}{\linewidth}
                        \begin{align*}
                            \intertext{By rationalisation}
                            \frac{(x+jy)}{(p+jq)} &= \frac{(x+jy) \times (p-jq)}{(p+jq) \times (p-jq)} = \frac{(x+jy)\times (p-jq)}{(p^2+q^2)}\\
                            \intertext{In polar form:}
                            \frac{Ae^{j\theta}}{Be^{j\phi}} &= \frac{A}{B}e^{j(\theta - \phi)} \tag{divide magnitudes, subtract angles}
                        \end{align*}
                        \caption{Division}
                    \end{subfigure}
                    \begin{subfigure}{\linewidth}
                        \begin{align*}
                            e^{j\left(\theta +2\pi n\right)} &= e^{j\theta}e^{j2\pi n}\\
                                                             &= e^{j\theta}\times 1\\
                                                             &= e^{j\theta} \tag{\forall n \in \mathrm{N}}
                        \end{align*}
                        \caption{Cyclicity}
                    \end{subfigure}
                \caption{Complex Number Operations}
                \end{figure}
        \section{Magnetic Circuits}
            \subsection{Phasors}
                \begin{flalign*}
                    v(t) = V_0 \cos(\omega t + \theta) = Re[Ve^{j\omega t}] = V_0 \angle \theta = V_0e^{j\theta}
                \end{flalign*}
                Only for sinusoids, component signals canot be represented as phasors (square, triangle waves, etc)\\
                \subsubsection{Solving RL RC Circuits}
                    \begin{flalign*}
                        v(t) &= L \odv{i}{t} + Ri(t)\\
                        Re\left[Ve^{j\omega t}\right] &= L \odv{}{t} Re\left[Ie^{j\omega t}\right]\\
                                                      &= LRe\left[I \odv{}{t}e^{j\omega t}\right]\\
                        Ve^{j\omega t} &= j\omega Ie^{j\omega t}\\
                        V &= j\omega LI\\
                    \end{flalign*}
                    Given an RL circuit with the following values
                    \begin{flalign}
                        v_s(t) &= 50\cos(100t)\\
                        C &= 0.2H\\
                        R &= 30\Omega
                    \end{flalign}
                    Using KCL
                    \begin{flalign*}
                        V_s &= V_R + V_L\\
                        &= IR + Ij\omega L\\
                        &= I\left(30 + \times j100\times 0.2\right)\\
                        50 &= I\left( 30 + j20\right)\\
                        \frac{50}{30 + j20} &= I\\
                        1.138 - j0.76923 &= I\\
                        1.386\angle -33.7\deg &= I\\
                        \therefore i(t) &= 1.386\cos(100t-0.588) [A]
                    \end{flalign*}
            \subsection{Oersted's Rules}
                \begin{itemize}
                    \item The magnetic field lines run encircle the current-carrying wire (Right hand screw rule)
                    \item The magnetic field lines lie in a plan perpendicular to the wire
                    \item If the direction of the current is reversed, the direction of the magnetic force reverses
                    \item The strength of a field is directly proportional to the magnitude of the current
                    \item The strength of the field at any point is inversely proportional to the distance of the point from the wire
                \end{itemize}
                This leads to Ampere's Law: The line integral of the magnetic field $B(x)$ around any closed curve $C$ is
                proportional to the total current $I$ passing through any surface bounded by the curve
                \begin{figure}[H]
                    \centering
                    \begin{flalign*}
                        \oint \mathbf{B} \cdot dl &= \mu_0 I
                    \end{flalign*}
                    \caption{Ampere's Law}
                \end{figure}
            \subsection{Electric Ohm's Law}
                
    \listoffigures
\end{document}:q
:
